\documentclass{article}

\usepackage{geometry, graphicx, amssymb, amsmath, amsthm, empheq, mdframed, booktabs, lipsum, graphicx, color, psfrag, pgfplots, bm}

\geometry{a4paper}

\newmdtheoremenv[linecolor=black,leftmargin=0pt,innerleftmargin=10pt,innerrightmargin=10pt,]{problem}{Problem}[]

\title{Assignment 1}
\author{Muhammad Meesum Ali Qazalbash}


\begin{document}
\maketitle

\begin{problem}[Distance between two adjacent maxima]
If \(\Delta y\) is the distance between two consecutive maxima, then they are given by
\begin{align*}
	\Delta y & = y_1 - y_2                                                \\
	         & = \frac{(m+1)\lambda D}{d} - \frac{m\lambda D}{d}          \\
	         & = \frac{\lambda D}{d}                                      \\
	         & = \frac{546 \text{ nm}\times 55\text{ cm}}{0.12\text{ mm}} \\
	         & = 2.50250 \text{ mm}
\end{align*}

\end{problem}

\begin{problem}[Angular seperation]
The angular seperation in double slit experiment is given by,
\[d\sin{\theta}=m\lambda\]
For small \(\theta\) we can approximate \(\sin{\theta}\approx\theta\), so we get the following equation,
\[d\theta=m\lambda\]
For 10\% more \(\theta\) we get,
\begin{align*}
	         & d\theta'=m\lambda'        \\
	\implies & d(1.1\theta)=m\lambda'    \\
	\implies & 1.1d\theta=m\lambda'      \\
	\implies & 1.1m\lambda=m\lambda'     \\
	\implies & 1.1\lambda=\lambda'       \\
	\implies & \lambda'= 647.9\text{ nm}
\end{align*}

\end{problem}

\newpage

\begin{problem}[Intensity]
\begin{enumerate}
	\item The electric field produce by these electromagnetic waves would be,
	      \begin{align*}
		      \mathbf{E} & = \mathbf{E}_1+\mathbf{E}_2+\mathbf{E}_3                                                                                                                                     \\
		                 & = \left(E_0\sin{\left(\omega t\right)}+E_0\sin{\left(\omega t+\phi\right)}+E_0\sin{\left(\omega t+2\phi\right)}\right)\hat{\mathbf{n}}                                       \\
		                 & = E_0\left(\sin{\left(\omega t\right)}+\sin{\left(\omega t+\phi\right)}+\sin{\left(\omega t+2\phi\right)}\right)\hat{\mathbf{n}}                                             \\
		                 & = E_0\left(\sin{\left(\omega t+\phi\right)}+\sin{\left(\omega t\right)}+\sin{\left(\omega t+2\phi\right)}\right)\hat{\mathbf{n}}                                             \\
		                 & = E_0\left(\sin{\left(\omega t+\phi\right)}+2\cos{\left(\frac{\omega t-\omega t-2\phi}{2}\right)}\sin{\left(\frac{\omega t+\omega t+2\phi}{2}\right)}\right)\hat{\mathbf{n}} \\
		                 & = E_0\left(\sin{\left(\omega t+\phi\right)}+2\cos{\left(\phi\right)}\sin{\left(\omega t+2\phi\right)}\right)\hat{\mathbf{n}}                                                 \\
		                 & = E_0\left(1+2\cos{\left(\phi\right)}\right)\sin{\left(\omega t+\phi\right)}\hat{\mathbf{n}}
	      \end{align*}

	      The magnitude square of the electric field is,
	      \[|\mathbf{E}|^2=E^2=E^2_0\left(1+2\cos{\left(\phi\right)}\right)^2\sin^2{\left(\omega t+\phi\right)}\]
	      Then the intesity would be.
	      \begin{align*}
		      I & = \left<E^2\right>                                                                                                            \\
		        & = E^2_0\left(1+2\cos{\left(\frac{2\pi d\sin{\theta}}{\lambda}\right)}\right)^2\left<\sin^2{\left(\omega t+\phi\right)}\right> \\
		        & = E^2_0\left(1+2\cos{\left(\frac{2\pi d\sin{\theta}}{\lambda}\right)}\right)^2
	      \end{align*}
	      \(I\) will be maximum when \(\phi=1\), so we get, \(I_0 = 9E^2_0\)
	      \[I = \frac{I_0}{9}\left(1+2\cos{\left(\frac{2\pi d\sin{\theta}}{\lambda}\right)}\right)^2\]
	\item The intensity is given by,
	      \[I = \frac{I_0}{9}\left(1+2\cos{\left(\phi\right)}\right)^2\]
	      Taking derivative of the intensity with respect to \(\phi\),
	      \begin{align*}
		      \frac{dI}{d\phi}                 & = \frac{d}{d\phi}\left(\frac{I_0}{9}\left(1+2\cos{\left(\phi\right)}\right)^2\right) \\
		      \frac{dI}{d\phi}                 & = -\frac{4I_0}{9}(1+2\cos{\left(\phi\right)})\sin{\left(\phi\right)}                 \\
		      \implies \frac{dI}{d\phi}        & = 0                                                                                  \\
		      \implies \sin{\left(\phi\right)} & = 0 \implies \phi_n = n\pi                                                           \\
		      \cos{\left(\phi\right)}          & = -\frac{1}{2} \implies \phi_n=\frac{2}{3}\pi+2n\pi                                  \\
	      \end{align*}
	      From second derivative test,
	      \begin{align*}
		      \frac{d^2I}{d\phi^2} & = \frac{d}{d\phi}\left(-\frac{4I_0}{9}(1+2\cos{\left(\phi\right)})\sin{\left(\phi\right)}\right) \\
		      \frac{d^2I}{d\phi^2} & = -\frac{4I_0}{9}\left(\cos{(\phi)}+2\cos{(2\phi)}\right)
	      \end{align*}
	      For \(\phi=\frac{2}{3}\pi+2n\pi\), we get,
	      \[\frac{d^2I}{d\phi^2} = -\frac{4I_0}{9}\left(\cos{\left(\frac{2}{3}\pi+2n\pi\right)}+2\cos{\left(2\left(\frac{2}{3}\pi+2n\pi\right)\right)}\right)\]
	      \[\frac{d^2I}{d\phi^2} = -\frac{4I_0}{9}\left(-\frac{1}{2}+2\left(-\frac{1}{2}\right)\right)\]
	      \[\frac{d^2I}{d\phi^2} = \frac{2I_0}{3}>0\]
	      Hence, \(\phi=\frac{2}{3}\pi+2n\pi\) is a minimum point. For \(\phi=\pi n\), we get,
	      \[\frac{d^2I}{d\phi^2} = -\frac{4I_0}{9}\left(\cos{\left(\pi n\right)}+2\cos{\left(2\pi n\right)}\right)\]
	      \[\frac{d^2I}{d\phi^2} = -\frac{4I_0}{9}\left((-1)^n+2\left(1\right)\right)\]
	      \[\frac{d^2I}{d\phi^2} = -\frac{4I_0}{9}\left((-1)^n+2\right)<0\]
	      Hence, \(\phi=\pi n\) is a maximum point. Primary maxima occurs at \(n=2k\) and secondary maxima occurs at \(n=2k+1\). Therefore the ratio of the intensity of the primary maxima to the secondary maxima is,
	      \[\frac{I_{\text{primary}}}{I_{\text{secondary}}} = \frac{\frac{I_0}{9}\left(1+2\cos{\left(2k\pi\right)}\right)^2}{\frac{I_0}{9}\left(1+2\cos{\left(\pi\left(2k+1\right)\right)}\right)^2}\]
	      \[\frac{I_{\text{primary}}}{I_{\text{secondary}}} = \frac{\left(1+2\right)^2}{\left(1-2\right)^2}\]
	      \[\frac{I_{\text{primary}}}{I_{\text{secondary}}} = 9\]
\end{enumerate}

\end{problem}

\end{document}