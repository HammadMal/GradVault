\documentclass[a4paper]{exam}

\usepackage[export]{adjustbox}
\usepackage{geometry}
\usepackage{graphicx}
\usepackage{hyperref}
\usepackage{mathtools}
\usepackage{tabularx}
\usepackage{titling}
\usepackage{listings}
\usepackage{amsmath}
\usepackage{amssymb}


\graphicspath{{images/}}

\printanswers

\title{Weekly Challenge 12: Dynamic Programming\\CS 412 Algorithms: Design and Analysis}
\author{team-2}  % <==== replace with your team name for grading
\date{Habib University | Spring 2023}

\runningheader{CS 412: Algorithms}{WC12: Dynamic Programming}{\theauthor}
\runningheadrule
\runningfootrule
\runningfooter{}{Page \thepage\ of \numpages}{}

\qformat{{\large\bf \thequestion. \thequestiontitle}\hfill}
\boxedpoints

\begin{document}
\maketitle

\begin{questions}


	\titledquestion{Shiddat-e-Milan}

	You have developed a novel approach to compute the compatibility among the users of your match-making portal, \href{https://habib.edu.pk}{Milaap}. The \textit{milan} score quantifies the compatibility between two users based on the content that they have consumed on your portal. There is a \textit{match}, as per your approach, when users consume not just the same content but in the same chronological order. For example, users who consumed the ``Shaam-e-Milaap'' page followed immediately by the ``Jashn-e-Milaap'' page match each other but not those who consumed the ``Fawaayed-e-Milaap'' page in between. The longer the match, the higher the milan score. Specifically, the milan score between two users is the largest number of pages that they have consumed in the same order.

	Given the chronologically sorted history of the consumption of pages by two users, you want to compute their milan score.
	\begin{parts}
		\part[3] Characterize the structure of an optimal solution.
		\part[2] Recursively define the value of an optimal solution.
		\part[3] Show how the value of an optimal solution can be computed in a bottom-up manner.
		\part[2] Argue about the time and space complexities of your approach.
	\end{parts}
	Remember to observe good attribution practices and cite any sources or references, \href{https://hulms.instructure.com/courses/2616/discussion_topics/29240}{especially if using AI}.

	\begin{solution}
		% Enter your solution here.
		\begin{parts}
			\part Let $X = \langle x_1, x_2,...x_n \rangle$ be a chronologically sorted history of the consumption of pages of user $X$ where each $x_i$ represent a page.
			Let $Y = \langle y_1, y_2,...y_m \rangle$ be a chronologically sorted history of the consumption of pages of user $Y$ where each $y_i$ represent a page.
			\\Then the optimal solution would be a sorted list of pages $Z = \langle z_1, z_2, ... z_k \rangle$ where each $z_i$ is a page consumed by both user $X$ and $Y$, and for each $1 \leq j\leq k-1$ both user $X$ and user $Y$ consumed page $z_{j+1}$ after page $z_j$,
			And if there exists a sorted list of pages $W = \langle w_1, w_2, ... w_l \rangle$ where each $w_i$ is a page consumed by both user $X$ and $Y$, and for each $1 \leq j\leq l-1$ both user $X$ and user $Y$ consumed page $w_{j+1}$ after page $w_j$, then $l \leq k$.
			\part The value of the value of the optimal solution would be the integer $k$ where $k$ is the largest number of pages user $X$ and $Y$ have consumed in the same order.
			Let $X = \langle x_1, x_2,...x_n \rangle$ be a chronologically sorted history of the consumption of pages of user $X$ where each $x_i$ represent a page.
			Let $Y = \langle y_1, y_2,...y_m \rangle$ be a chronologically sorted history of the consumption of pages of user $Y$ where each $y_i$ represent a page.
			We define $V(i,j)$ as the length of the longest common sequence of pages in chronological order from the sorted history $X_i = \langle x_1, x_2,...x_i \rangle$ and $Y_j = \langle y_1, y_2,...y_j\rangle$, for $i \leq n$ and $j \leq m$.
			\\More formally for sorted history $X_i = \langle x_1, x_2,...x_i \rangle$ and $Y_j = \langle y_1, y_2,...y_j\rangle$, for $i \leq n$ and $j \leq m$, there exists $Z_{ij} = \langle z_1, z_2, ... z_k \rangle$ where each $z_a$ belongs to list $X_i$ and $Y_j$, and for each $1 \leq b\leq k-1$ both user $X$ and user $Y$ consumed page $z_{b+1}$ after page $z_b$,
			And if there exists a sorted list of pages $W_{ij} = \langle w_1, w_2, ... w_l \rangle$ where each $w_a$ belongs to both lists $X_i$ and $Y_j$, and for each $1 \leq b\leq l-1$ both user $X$ and user $Y$ consumed page $w_{b+1}$ after page $w_b$, then $l \leq k$. Then $V(i,j) = k$.
			\\We can compute $V(i,j)$ recursively as follows:
			$$V(i,j) = \begin{cases}
					0               & \text{ if } i = 0 \lor j = 0 \lor  x_i \neq y_j \\
					V(i-1, j-1) + 1 & \text{otherwise}
				\end{cases}$$
			Then for chronologically sorted history of the consumption of pages $X$ and $Y$ the value of the optimal solution $V$ is conputed as follows:
			$$V = \max_{0 \leq i \leq n, 0 \leq j \leq m} V(i,j)$$
			\part To compute the value of the optimal solution in a bottom up manner in the following way.
			\\We consturct an $n \times m$ array $M$, where in each $M[i][j]$ entry of $M$ we will store the value of $V(i,j)$.
			\\We will start populating the array row-wise starting from row $0$, we will also keep track of the maximum value of $V(i,j)$, for $0\leq i \leq n$ and $0\leq j \leq m$.
			\\For $0 < i \leq n$ and $0 < j \leq m$ when we compute $V(i,j)$ it may need to look up $V(i-1,j-1)$, which will already be store in the matrix so we won't need to compute it again.
			\\The following python code computes the value of the optimal solution in a bottom up manner.
			\lstinputlisting[language=python]{milan.py}
			\part The algorithm first create an $n \times m$ array which takes $O(n\times m)$ time and then iterate through the matrix populating it row by row, at each $M[i][j]$ entry of the matrix,
			the algorithm checks the value of $i$ and $j$, may check the values of $X[i]$ and $Y[j]$ and may check the value of $X[i-1]$ and $Y[j-1]$, which are constant time operations.
			\\Therefore as we first do some $O(n\times m)$ work then do some constant amount of work $O(n\times m)$ times, the time complexity of the algorithm is $O(n\times m) + O(C \times n\times m)$ where $C \in \mathbb{R}$ is a constant.
			\\So asymptotically the time complexity is $O(n\times m)$.
			\\For the space complexity we create an $n \times m$ array which takes some $O(n \times m)$ space, besides that we need some constant space so asymptotically the space complexity is $O(n \times m)$.
		\end{parts}
	\end{solution}

\end{questions}

\end{document}

%%% Local Variables:
%%% mode: latex
%%% TeX-master: t
%%% End:
