\documentclass[a4paper, 11pt]{article}
\usepackage[pdftex]{graphicx}
\usepackage{parskip}
\usepackage{hyperref}
\usepackage[all]{hypcap}
\usepackage{amsmath}
\usepackage{amssymb}
\usepackage{amsfonts}
\usepackage{enumitem}
\usepackage{mathtools}
\usepackage{graphicx}
\usepackage{pgfplots}

\title{Habib University\\Math 310 - Introduction to Probability and Statistics\\Assignment 2}
\author{Muhammad Meesum Ali Qazalbash}


\newcommand{\mat}[1]{\boldsymbol { \mathsf{#1}} }
\pgfplotsset{compat=1.18}
\begin{document}
\setlength{\parskip}{10pt}
\setlength{\parindent}{0pt}
\DeclareGraphicsExtensions{.pdf,.png,.gif,.jpg}
\maketitle


\section*{Question 1} In my gloves drawer there are 21 red gloves, 8 blue gloves, and 4 green gloves. The gloves are not stacked in pairs. If I randomly pull out 6 gloves, what is the probability that I pull out exactly 2 gloves of each color?
\section*{Solution}
% In this random experiment as we pull out 6 gloves, there are exactly 2 gloves of each color. Question does not mention any kind of proiorty to the order of the gloves, then the event of picking gloves is a binomial event. Let $\Omega$ be the sample space with $X$ as a discrete binomial random variable over this set.
In this random experiment as we pull out 6 gloves, there are exactly 2 gloves of each color. Question does not mention any kind of proiorty to the order of the gloves. Let $\Omega$ be the sample space with $X$ as a discrete random variable over this set.
\begin{align*}
    \Omega_{r}:= & \{r_{i}:i=1,2,3,\cdots,21\}            \\
    \Omega_{b}:= & \{b_{i}:i=1,2,3,\cdots,8\}             \\
    \Omega_{g}:= & \{g_{i}:i=1,2,3,4\}                    \\
    \Omega:=     & \Omega_{r}\cup\Omega_{b}\cup\Omega_{g}
\end{align*}
% There are sub DRVs on the sub sample space which are also binomial in nature. The required probability can be calculated as,
% \[P(X_{r}=2\wedge X_{b}=2\wedge X_{g}=2)=P(X_{r}=2)P(X_{b}=2)P(X_{g}=2)\]
% \[=\left(\binom{21}{2}\left(\frac{2}{21}\right)^{2}\left(\frac{19}{21}\right)^{19}\right)\left(\binom{8}{2}\left(\frac{2}{8}\right)^{2}\left(\frac{6}{8}\right)^{6}\right)\left(\binom{4}{2}\left(\frac{2}{4}\right)^{2}\left(\frac{2}{4}\right)^{2}\right)\]
% \[=(0.28444\cdots)(0.31146\cdots)(0.375)\]
% \[P(X_{r}=2\wedge X_{b}=2\wedge X_{g}=2)=0.03322\]
The required probability can be calculated as,
\begin{equation*}
    \begin{split}
        P(X_{r}=2\wedge X_{b}=2\wedge X_{g}=2)&=\frac{\binom{21}{2}\binom{8}{2}\binom{4}{2}}{\binom{33}{6}}\\
        &=\frac{35280}{1107568}\\
        &=\frac{315}{9889}
    \end{split}
\end{equation*}
\section*{Question 2} A pizza place offers the choice of the following toppings: extra cheese, mushrooms, pineapple, chicken, fish, onions, and green peppers. Assume that the order of toppings is irrelevant. Also assume that toppings cannot be reused
\begin{enumerate}[label=(\alph*)]
    \item How many 3-topping pizzas are possible?
    \item If I randomly order a 3-topping pizza, what is the probability that fish is a topping?
    \item If I randomly order a 3-topping pizza, what is the probability that fish and chicken are toppings?
\end{enumerate}
\section*{Solution}
Let the set of toppings be $T$. The total number of possible 3-topping can be found out by using combination because we have to choose 3 toppings out of 7 choices and order does not matter.
\begin{enumerate}[label=(\alph*)]
    \item \[\Omega:=\big\{\{i,j,k\}:i,j,k\in T\wedge i\ne j\ne k\big\}\]
          \[|\Omega|=\binom{7}{3}=35\]
    \item Probability of getting a fish topping while odering a 3-topping pizza is,
          \begin{align*}
              P(\text{fish is a topping}) & = \frac{\binom{1}{1}\binom{6}{2}}{\binom{7}{3}} \\
              P(\text{fish is a topping}) & = \frac{3}{7}
          \end{align*}
    \item Probability of getting a fish and chicken topping while odering a 3-topping pizza is,
          \begin{align*}
              P(\text{fish and chicken are topping}) & = \frac{\binom{1}{1}\binom{1}{1}\binom{5}{1}}{\binom{7}{3}} \\
              P(\text{fish and chicken are topping}) & = \frac{1}{7}
          \end{align*}
\end{enumerate}
\break
\section*{Question 3} One out of every eight calls to your house is a telemarketer. Assume that the telemarketers' calls are independent. Let X denote the number of telemarketers during the next three calls.
\begin{enumerate}[label=(\alph*)]
    \item What is the mass of X?
    \item Draw the PMF of X.
\end{enumerate}
\section*{Solution}
The probability of call of telemarketer is $p=\frac{1}{8}$. Calling of telemarketer is a binomial event.
\begin{enumerate}[label=(\alph*)]
    \item The mass of the random variable $X$ is,
          \[P(X=i)=\binom{3}{i}\left(\frac{1}{8}\right)^{i}\left(1-\frac{1}{8}\right)^{3-i}\qquad i\in\{0,1,2,3\}\]
          \begin{equation*}
              \begin{split}
                  P(X=0)=\frac{343}{512}\qquad P(X=1)=\frac{147}{512}\\
                  P(X=2)=\frac{21}{512}\qquad P(X=3)=\frac{1}{512}
              \end{split}
          \end{equation*}
    \item The graph of $P(X=x)$ will be,
          \begin{center}
              \begin{tikzpicture}
                  \begin{axis} [
                          title = $X$ : number of telemarketers during the next three calls,
                          ybar,
                          ymin=0,
                          %   ymax=1,
                          %   nodes near coords,
                          %   nodes near coords align={vertical},
                          %   height=10cm,
                          %   width=12cm,
                          xtick=data,
                          ylabel={$P[X=x]$},
                          xlabel={$x$}
                      ]
                      \addplot coordinates {
                              (0,343/512)
                              (1,147/512)
                              (2,21/512)
                              (3,1/512)
                          };
                      \legend {X}
                  \end{axis}
              \end{tikzpicture}
          \end{center}
\end{enumerate}
\section*{Question 4} One out of every eight calls to your house is a telemarketer. Assume that the telemarketers' calls are independent. Let X denote the number of callers until the telemarketer calls. If $n$ is a non-negative integer, What is $P(X\geq n)$?
\section*{Solution}
The random variable is a Geometric random variable, because we have to check until the first success appears i.e. call of telemarketer. The probability of call of telemarketer is $p=\frac{1}{8}$, then we can easily find the probability of number of calls it take untill the telemarketer makes a call.
\begin{equation*}
    \begin{split}
        P(X=k)&=\left(1-p\right)^{k-1}p\\
        &=\left(1-\frac{1}{8}\right)^{k-1}\left(\frac{1}{8}\right)\\
        &=\left(\frac{7}{8}\right)^{k-1}\left(\frac{1}{8}\right)\\
        P(X=k)&=\frac{7^{k-1}}{8^{k}}
    \end{split}
\end{equation*}
The probability of number of class more than $n$ is,
\begin{equation*}
    \begin{split}
        P(X\ge n)&=\sum_{k=0}^{\infty}P(X=n+k)\\
        &=\sum_{k=0}^{\infty}\frac{7^{n+k-1}}{8^{n+k}}\\
        &=\frac{7^{n-1}}{8^{n}}\sum_{k=0}^{\infty}\left(\frac{7}{8}\right)^{k}\\
        &=\frac{7^{n-1}}{8^{n}}\frac{1}{1-\frac{7}{8}}\\
        P(X\ge n)&=\left(\frac{7}{8}\right)^{n-1}
    \end{split}
\end{equation*}

\break
\section*{Question 5} Suppose we draw 5 cards at random, without replacement, from a deck of 52 cards (such a deck includes 4 Queens). Let $X$ denote the number of Queens drawn. Define some indicator random variables $X_1$,$\cdots$, $X_5$ so that $X = X_1 +\cdots + X_5$. Then use the random variables you created to find $E(X)$. Also calculate the $E[X]$ if this time 5 cards are drawn with replacement.
\section*{Solution}
Let the indicator random variable be defined as,
\[
    X_{j}= \begin{cases}
        1 & j\text{th card is Queen} \\
        0 & \text{else}
    \end{cases}
\]
The expectation of $X$ will be,
\begin{equation*}
    \begin{split}
        E[X]=&E\left[\sum_{j=1}^{5}X_{j}\right]\\
        E[X]=&\sum_{j=1}^{5}E[X_{j}]\\
        =&\sum_{j=1}^{5}\left(\left(\frac{4}{52}\right)(1)+(\frac{48}{52})(0)\right)\\
        =&\frac{1}{13}\sum_{j=1}^{5}1\\
        E[X]=&\frac{5}{13}
    \end{split}
\end{equation*}
In case of replacement the probabilities will remain same. Therefore, the expectation will also remain same.
\[\therefore E[X]=\frac{5}{13}\]

\break
\section*{Question 6} Suppose that a drawer contains 8 marbles: 2 are red, 2 are blue, 2 are green, and 2 are yellow. The marbles are rolling around in a drawer, so that all possibilities are equally likely when they are drawn. Alice chooses 2 marbles without replacement, and then Bob chooses
2 marbles. Let X denote the number of red marbles that are chosen altogether (if Alice and Bob put their collected marbles together after picking).
\begin{enumerate}[label=(\alph*)]
    \item Find $P(X = 0)$.
    \item Find $P(X = 1)$.
    \item Find $P(X = 2)$.
\end{enumerate}
\section*{Solution}
We can address this problem, if we focus on the total sum Alice and Bob getting. If total red marbles are 0 then they have taken out 4 marbles from subset of sample space with no red marbles.
\[P(X=0)=\frac{\binom{6}{4}}{\binom{8}{4}}=\frac{3}{14}\]
If total red marbles are 1 then they have taken out 1 marble out of 2 red marbles and 3 marbles from subset of sample space with no red marbles.
\[P(X=1)=\frac{\binom{2}{1}\binom{6}{3}}{\binom{8}{4}}=\frac{4}{7}\]
If total red marbles are 2 then they have taken out 2 marble out of 2 red marbles and 2 marbles from subset of sample space with no red marbles.
\begin{equation*}
    \begin{split}
        P(X=2)=&\frac{\binom{2}{2}\binom{6}{2}}{\binom{8}{4}}\\
        =&\frac{3}{14}
    \end{split}
\end{equation*}

\break
\section*{Question 7} Suppose you roll 3 fair six sided dice. Draw probability mass function for the following random variables.
\begin{enumerate}[label=(\alph*)]
    \item $X$ represents the sum of dice.
    \item $Y$ represents the maximum of three dice.
    \item $Z$ represents the value of the first die rolled.
\end{enumerate}
\section*{Solution}
\begin{center}
    \begin{tikzpicture}
        \begin{axis} [
                title = $X$ : sum of dice,
                ybar,
                ymin=0,
                % ymax=1,
                % height=10cm,
                width=11cm,
                xtick=data,
                ylabel={$P[X=x]$},
                xlabel={$x$}
            ]
            \addplot coordinates {
                    (3,1/216)
                    (4,3/216)
                    (5,6/216)
                    (6,10/216)
                    (7,15/216)
                    (8,21/216)
                    (9,25/216)
                    (10,27/216)
                    (11,27/216)
                    (12,25/216)
                    (13,21/216)
                    (14,15/216)
                    (15,10/216)
                    (16,6/216)
                    (17,3/216)
                    (18,1/216)
                };

            \addplot[red,sharp plot,update limits=false]
            coordinates {(0,1/8) (20,1/8)}
            node[above] at (axis cs:4,27/216) {max : 0.125};

            \addplot[red,sharp plot,update limits=false]
            coordinates {(0,1/216) (20,1/216)}
            node[above] at (axis cs:4,1/216) {min : 0.00463};
            \legend {X}
        \end{axis}
    \end{tikzpicture}
    \begin{tikzpicture}
        \begin{axis} [
                title =$Y$ : maximum of three dice,
                legend pos=north west,
                ybar,
                ymin=0,
                % ymax=1,
                % % height=10cm,
                width=11cm,
                xtick=data,
                ylabel={$P[Y=y]$},
                xlabel={$y$}
            ]
            \addplot [style = {red, fill = pink}]
            coordinates {
                    (1,1/216)
                    (2,7/216)
                    (3,19/216)
                    (4,37/216)
                    (5,61/216)
                    (6,91/216)
                };
            \addplot[violet,sharp plot,update limits=false]
            coordinates {(0,91/216) (7,91/216)}
            node[above] at (axis cs:4,91/216) {max : 0.4213};

            \addplot[violet,sharp plot,update limits=false]
            coordinates {(0,1/216) (7,1/216)}
            node[above] at (axis cs:2,1/216) {min : 0.00463};
            \legend {Y}
        \end{axis}
    \end{tikzpicture}
    \begin{tikzpicture}
        \begin{axis} [
                title=$Z$ : value of the first die rolled,
                ybar,
                ymin=0,
                % ymax=1,
                % height=10cm,
                width=11cm,
                xtick=data,
                ylabel={$P[Z=z]$},
                xlabel={$z$}
            ]
            \addplot [style = {orange, fill = yellow}]
            coordinates {
                    (1,36/216)
                    (2,36/216)
                    (3,36/216)
                    (4,36/216)
                    (5,36/216)
                    (6,36/216)
                };
            \addplot[red,sharp plot,update limits=false]
            coordinates {(0,1/6) (7,1/6)}
            node[above] at (axis cs:2,1/6) {min : max : 0.1667};
            \legend {Z}
        \end{axis}
    \end{tikzpicture}
\end{center}

\end{document}