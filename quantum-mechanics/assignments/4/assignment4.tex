\documentclass[addpoints]{exam}

\usepackage{geometry}
\usepackage{amssymb}
\usepackage{amsmath}
\usepackage{amsthm}
\usepackage{empheq}
\usepackage{mdframed}
\usepackage{graphicx}
\usepackage{color}
\usepackage{psfrag}
\usepackage{pgfplots}
\usepackage{bm}
\usepackage{hyperref}
\usepackage[backend=bibtex,
			style=alphabetic,
			sorting=ynt]{biblatex}
\addbibresource{reference.bib}

\geometry{a4paper}

\definecolor{ocre}{RGB}{243,102,25}
\definecolor{mygray}{RGB}{243,243,244}
\definecolor{deepGreen}{RGB}{26,111,0}
\definecolor{shallowGreen}{RGB}{235,255,255}
\definecolor{deepBlue}{RGB}{61,124,222}
\definecolor{shallowBlue}{RGB}{235,249,255}

\newcommand\orangebox[1]{\fcolorbox{ocre}{mygray}{\hspace{1em}#1\hspace{1em}}}

\newtheoremstyle{mytheoremstyle}{3pt}{3pt}{\normalfont}{0cm}{\rmfamily\bfseries}{}{1em}{{\color{black}\thmname{#1}~\thmnumber{#2}}\thmnote{\,--\,#3}}
\newtheoremstyle{myproblemstyle}{3pt}{3pt}{\normalfont}{0cm}{\rmfamily\bfseries}{}{1em}{{\color{black}\thmname{#1}~\thmnumber{#2}}\thmnote{\,--\,#3}}
\theoremstyle{mytheoremstyle}
\newmdtheoremenv[linewidth=1pt,backgroundcolor=shallowGreen,linecolor=deepGreen,leftmargin=0pt,innerleftmargin=20pt,innerrightmargin=20pt,]{theorem}{Theorem}[section]
\theoremstyle{mytheoremstyle}
\newmdtheoremenv[linewidth=1pt,backgroundcolor=shallowBlue,linecolor=deepBlue,leftmargin=0pt,innerleftmargin=20pt,innerrightmargin=20pt,]{definition}{Definition}[section]
\theoremstyle{myproblemstyle}
\newmdtheoremenv[linecolor=black,leftmargin=0pt,innerleftmargin=10pt,innerrightmargin=10pt,]{problem}{Problem}[section]

\usepgfplotslibrary{colorbrewer}
\pgfplotsset{width=8cm,compat=1.9}
\printanswers

\title{QM Assignment 4}
\author{Muhammad Meesum Ali Qazalbash}

\begin{document}
\maketitle

\begin{questions}
	\question[15] Let's consider one spatial dimension and time. Remember, we have defined the position and momentum operators in the class, \(\hat{x}\) and \(\hat{p}\), respectively. The momentum operator is the derivative operator defined as:
	\[\hat{p}=-i\hslash\frac{d}{dx}\]
	where \(\hslash\) is the modified Planck's constant. You now know that if \(\psi(x, t)\) is the wavefunction of the quantum system, then \(|\psi(x, t)|^2\) is the probability distribution function. For any operator \(\hat{Q}\), the expectation of it is defined by the following equation:
	\[\left\langle\hat{Q}\right\rangle = \int \psi^{*}(x,t)\hat{Q}\psi(x,t)dx\]
	\begin{enumerate}
		\item Calculate the expectations of the position and momentum operators, \(\langle\hat{x}\rangle\) and \(\langle\hat{p}\rangle\).
		\item Calculate the time derivative of the position expectation, \(\displaystyle\frac{d\langle\hat{x}\rangle}{dt}\).
		\item We define the time derivative of the position expectation by \(\hat{v}\). Show that \(\langle p\rangle = m\langle v\rangle\).
		\item The kinetic energy is defined as \(\displaystyle T = \frac{p}{2m}\) . Define the kinetic energy operator and calculate the expectation value of the kinetic energy operator \(\langle\hat{T}\rangle\).
		\item Show that \[\frac{d\langle\hat{p}\rangle}{dt}=\left\langle-\frac{dV}{dx}\right\rangle\] This is known as Ehrenfest's Theorem, which shows that the expectation values obey Newton's Second law.
	\end{enumerate}
	\begin{solution}
		\begin{enumerate}
			\item The expectation of the position and momentum operator are,
			      \begin{equation*}
				      \begin{aligned}
					      \left\langle\hat{x}\right\rangle & = \int \psi^{*}\hat{x}\psi dx      \\
					                                       & = \int \psi^{*}x\psi       dx      \\
					                                       & = \int x|\psi|^2                dx \\
					                                       & = \langle x\rangle
				      \end{aligned}
			      \end{equation*}
			      \begin{equation*}
				      \begin{aligned}
					      \left\langle\hat{p}\right\rangle & = \int \psi^{*}\hat{p}\psi                            dx \\
					                                       & = \int \psi^{*}\left(-i\hslash\frac{d}{dx}\right)\psi dx \\
					                                       & = -i\hslash\int \psi^{*} d\psi
				      \end{aligned}
			      \end{equation*}
			\item The time derivative of the position expectation is,
			      \begin{equation*}
				      \begin{aligned}
					      \frac{d\langle\hat{x}\rangle}{dt} & = \frac{d}{dt}\int x|\psi|^2                             dx \\
					      \frac{d\langle\hat{x}\rangle}{dt} & = \int \frac{\partial}{\partial t}\left(x|\psi|^2\right) dx \\
					      \frac{d\langle\hat{x}\rangle}{dt} & = \int x\frac{\partial}{\partial t}\left(|\psi|^2\right)dx
				      \end{aligned}
			      \end{equation*}
			      We from \cite{DavidJ.Griffiths:2005:IntroductiontoQuantumMechanics}
			      \begin{equation*}
				      \begin{aligned}
					      \frac{\partial}{\partial t}|\psi|^2        & = \frac{i\hslash}{2m}\frac{\partial}{\partial x}\left(\psi^{*}\frac{\partial\psi}{\partial x}-\frac{\partial\psi^{*}}{\partial x}\psi\right)          \\
					      \implies \frac{d\langle\hat{x}\rangle}{dt} & = \int x\frac{i\hslash}{2m}\frac{\partial}{\partial x}\left(\psi^{*}\frac{\partial\psi}{\partial x}-\frac{\partial\psi^{*}}{\partial x}\psi\right) dx \\
					      \implies \frac{d\langle\hat{x}\rangle}{dt} & = \frac{i\hslash}{2m}\int x\frac{\partial}{\partial x}\left(\psi^{*}\frac{\partial\psi}{\partial x}-\frac{\partial\psi^{*}}{\partial x}\psi\right) dx \\
				      \end{aligned}
			      \end{equation*}
			      By using integration by parts we come to point where our expression looks like\cite{DavidJ.Griffiths:2005:IntroductiontoQuantumMechanics},
			      \[\frac{d\langle\hat{x}\rangle}{dt} = -\frac{i\hslash}{m}\int \psi^{*}\frac{\partial\psi}{\partial x}dx \]
			\item We get that,
			      \begin{equation*}
				      \begin{aligned}
					      \langle p\rangle & = -i\hslash\int \psi^{*} d\psi                         \\
					      \langle p\rangle & = m\left(-\frac{i\hslash}{m}\int \psi^{*} d\psi\right) \\
					      \langle p\rangle & = m\frac{d\langle x\rangle}{dt}                        \\
					      \langle p\rangle & = m\langle v\rangle
				      \end{aligned}
			      \end{equation*}

			      \newpage

			\item The kinetic energy is defined as \(\displaystyle T = \frac{p}{2m}\) . Define the kinetic energy operator and calculate the expectation value of the kinetic energy operator \(\langle\hat{T}\rangle\).
			      \begin{equation*}
				      \begin{aligned}
					      \hat{T}               & = \frac{\hat{p}^2}{2m}                                                                   \\
					      \hat{T}               & = \frac{1}{2m}\left(-i\hslash\frac{\partial}{\partial x}\right)^2                        \\
					      \hat{T}               & = -\frac{\hslash^2}{2m}\frac{\partial^2}{\partial x^2}                                   \\
					      \langle\hat{T}\rangle & = \int \psi^{*} \left(-\frac{\hslash^2}{2m}\frac{\partial^2}{\partial x^2}\right)\psi dx \\
					      \langle\hat{T}\rangle & = -\frac{\hslash^2}{2m}\int \psi^{*}\frac{\partial^2\psi}{\partial x^2} dx               \\
				      \end{aligned}
			      \end{equation*}
			\item The Ehrenfest Theorem\cite{ET} says that the expectation values obey Newton's Second law. We have,
			      \begin{equation*}
				      \begin{aligned}
					      \frac{d\langle\hat{p}\rangle}{dt} & = -i\hslash\frac{d}{dt}\int \psi^{*}\frac{\partial \psi}{\partial x}dx                                                                                                 \\
					      \frac{d\langle\hat{p}\rangle}{dt} & = -i\hslash\int \frac{\partial}{\partial t}\left(\psi^{*}\frac{\partial \psi}{\partial x}\right)dx                                                                     \\
					      \frac{d\langle\hat{p}\rangle}{dt} & = -i\hslash\int \left(\frac{\partial \psi^{*}}{\partial t}\frac{\partial \psi}{\partial x}+\psi^{*}\frac{\partial}{\partial t}\frac{\partial\psi}{\partial x}\right)dx
				      \end{aligned}
			      \end{equation*}
			      Here we have integrated by parts. Substituting from Schrödinger's equation, and simplifying, we obtain,
			      \begin{equation*}
				      \begin{aligned}
					      \frac{d\langle\hat{p}\rangle}{dt} & = \int \left(-\frac{\hslash^{2}}{2m}\frac{\partial}{\partial x}\left(\frac{\partial \psi^{*}}{\partial x}\frac{\partial \psi}{\partial t}\right)+V(x)\frac{\partial|\psi|^2}{\partial x}\right)dx \\
					      \frac{d\langle\hat{p}\rangle}{dt} & = \int V(x)\frac{\partial|\psi|^2}{\partial t}dx                                                                                                                                                  \\
					      \frac{d\langle\hat{p}\rangle}{dt} & = -\int \frac{\partial V}{\partial x}|\psi|^2dx                                                                                                                                                   \\
					      \frac{d\langle\hat{p}\rangle}{dt} & = -\left\langle\frac{\partial V}{\partial x}\right\rangle
				      \end{aligned}
			      \end{equation*}


		\end{enumerate}
	\end{solution}

	\question[15] In class, we derived the Probability conservation law in QM. The probability conservation tells us that the particle is conserved "locally" and is stable. Suppose you want to describe an "unstable" particle that spontaneously disintegrates with a lifetime of \(\tau\). In that case, the total probability of finding the particle somewhere should not be constant but decrease exponentially: \[P(t)\equiv \int |\psi(x,t)|^2 dx=e^{-t/\tau}\] In our derivation, we used the fact that the potential energy \(V\) is real. This leads to the conservation of probability. What if we assign to \(V\) an imaginary part: \[V=V_0-i\Gamma\] where \(V_0\) is the true potential energy and \(\Gamma\) is the positive real constant.
	\begin{enumerate}
		\item Calculate \(\displaystyle\frac{dP}{dt}\).
		\item Solve for \(P(t)\) and find the lifetime of the particle in term of \(\Gamma\).
	\end{enumerate}
	\begin{solution}
		\begin{enumerate}
			\item We have,
			      \begin{equation*}
				      \begin{aligned}
					      \frac{dP}{dt} & = \int \frac{\partial}{\partial t}|\psi|^2dx                                                          \\
					                    & = \int \frac{\partial}{\partial t}\left(\psi\psi^{*}\right)dx                                         \\
					                    & = \int \left(\frac{\partial\psi}{\partial t}\psi^{*}+\frac{\partial\psi^{*}}{\partial t}\psi\right)dx
				      \end{aligned}
			      \end{equation*}
			      From Schrödinger equation we have,
			      \begin{equation*}
				      \begin{aligned}
					      \frac{\partial\psi}{\partial t}     & = -\frac{i\hslash}{2m}\frac{\partial^2\psi}{\partial x^2}+\frac{i}{\hslash}V_0\psi+\frac{\Gamma}{\hslash}\psi            \\
					      \frac{\partial\psi^{*}}{\partial t} & = \frac{i\hslash}{2m}\psi^{*}\frac{\partial^2}{\partial x^2}-\psi^{*}\frac{i}{\hslash}V_0+\psi^{*}\frac{\Gamma}{\hslash}
				      \end{aligned}
			      \end{equation*}
			      Therefore we get,
			      \begin{equation*}
				      \begin{aligned}
					      \frac{\partial\psi}{\partial t}\psi^{*}                                         & = \left(-\frac{i\hslash}{2m}\frac{\partial^2\psi}{\partial x^2}+\frac{i}{\hslash}V_0\psi+\frac{\Gamma}{\hslash}\psi\right)\psi^{*}        \\
					      \frac{\partial\psi^{*}}{\partial t}\psi                                         & = \left(\frac{i\hslash}{2m}\psi^{*}\frac{\partial^2}{\partial x^2}-\psi^{*}\frac{i}{\hslash}V_0+\psi^{*}\frac{\Gamma}{\hslash}\right)\psi \\
					      \hline                                                                                                                                                                                                                      \\
					      \frac{\partial\psi}{\partial t}\psi^{*}+\frac{\partial\psi^{*}}{\partial t}\psi & = -\frac{1}{i\hslash}\left(-i\Gamma\psi\psi^{*}\right)+\frac{1}{i\hslash}\psi^{*}\left(i\Gamma\psi\right)                                 \\
					      \frac{dP}{dt}                                                                   & = \int \left(-\frac{1}{i\hslash}\left(-i\Gamma\psi\psi^{*}\right)+\frac{1}{i\hslash}\psi^{*}\left(i\Gamma\psi\right)\right)dx             \\
					                                                                                      & = \int \frac{2\Gamma}{\hslash}|\psi|^2dx                                                                                                  \\
					                                                                                      & = \frac{2\Gamma}{\hslash}\int |\psi|^2dx                                                                                                  \\
					                                                                                      & = \frac{2\Gamma}{\hslash}P                                                                                                                \\
				      \end{aligned}
			      \end{equation*}
			\item Now the probability conservation law is no longer valid. The probability of finding the particle decreases exponentially with time. The lifetime of the particle is given by,
			      \begin{equation*}
				      \begin{aligned}
					      \frac{dP}{dt} = \frac{2\Gamma}{\hslash}P & \implies -\frac{1}{\tau}e^{-t/\tau} = \frac{2\Gamma}{\hslash}e^{-t/\tau} \\
					                                               & \implies \tau = -\frac{\hslash}{2\Gamma}
				      \end{aligned}
			      \end{equation*}
		\end{enumerate}
	\end{solution}
\end{questions}

\printbibliography


\end{document}