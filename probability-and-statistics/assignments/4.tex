\documentclass[answers]{exam}

%% Language and font encodings
\usepackage[english]{babel}
\usepackage[utf8x]{inputenc}
\usepackage[T1]{fontenc}
% \usepackage{enumitem}
%% Sets page size and margins
\usepackage[a4paper,margin=2cm]{geometry}

%% Useful packages
\usepackage{amsmath}
\usepackage{amssymb}
\usepackage{graphicx}
\usepackage{paralist}
\usepackage{framed}
\usepackage{tikz}
\usepackage{float}
\usepackage{amsthm}
\newtheorem{theorem}{Theorem}
\tikzset{
  % define the bar graph element
  bar/.pic={
    \fill (-.1,0) rectangle (.1,#1) (0,#1) node[above,scale=1/2]{$#1$};
  }
}
\usetikzlibrary{matrix}

\setlength\FrameSep{4pt}
\title{Probability \& Statistics\\ Assignment 4}
\author{Muhammad Meesum Ali Qazalbash, 06861}
\date{\today{}}
\begin{document}
\maketitle

\begin{questions}


    \question[35]{
        Consider two continuous random variables X and Y with joint p.d.f.
        \begin{equation*}
            f_{X,Y}(x,y) = \begin{cases}
                \frac{2}{81}x^2 y & 0<x<K \;,\; 0 < y < K \\
                0                 & \mbox{Otherwise}
            \end{cases}
        \end{equation*}

        \begin{enumerate}[(a)]
            \item Find K so that $f(x,y)$ becomes a valid joint p.d.f.
            \item Are X and Y independent? (Provide detailed steps)
        \end{enumerate}

    }

    \begin{framed}
        \begin{enumerate}[(a)]
            \item Any valid pdf will normalize to 1.
                  \begin{equation*}
                      \begin{split}
                          \int_{-\infty}^{\infty}\int_{-\infty}^{\infty}f_{X,Y}(x,y)\operatorname{d}x\operatorname{d}y & = 1\\
                          \int_{0}^{K}\int_{0}^{K}\frac{2}{81}x^2y\operatorname{d}x\operatorname{d}y & = 1\\
                          \frac{2}{81}\int_{0}^{K}x^2\operatorname{d}x\int_{0}^{K}y\operatorname{d}y & = 1\\
                          \frac{2}{81}\frac{K^3}{3}\frac{K^2}{2} & = 1\\
                          K^5 & = 243\\
                          K & = 3
                      \end{split}
                  \end{equation*}
            \item By the definition of independence,
                  \[f_{X,Y}(x,y)=f_X(x)f_Y(y)\]
                  The marginals are,
                  \begin{equation*}
                      \begin{split}
                          f_X(x) & = \int_{0}^{3}\frac{2}{81}x^2 y\operatorname{d}y\\
                          & = \frac{2}{81}x^2\int_{0}^{3}y\operatorname{d}y\\
                          & = \frac{2}{81}x^2\left(\frac{3^2}{2}-\frac{0^2}{2}\right)\\
                          f_X(x) & = \frac{x^2}{27}\\
                          f_Y(y) & = \int_{0}^{3}\frac{2}{81}x^2y\operatorname{d}x\\
                          & = \frac{2}{81}\int_{0}^{3}x^2\operatorname{d}x\int_{c}^{d}y\operatorname{d}y\\
                          & = \frac{2}{81}y\left(\frac{3^3}{3}-\frac{0^3}{3}\right)\\
                          f_Y(y) & = \frac{2y}{9}
                      \end{split}
                  \end{equation*}
                  We can see that,
                  \[f_{X,Y}(x,y)=f_X(x)f_Y(y)\]
                  Hence, $X$ and $Y$ are independent random variable.
        \end{enumerate}
    \end{framed}

    \break
    \question[35]{
        Let X and Y be jointly continuous random variables with joint PDF:
        \begin{equation*}
            f_{X,Y}(x,y) = \begin{cases}
                6 e^{-2x-3y} & x,y\geq 0        \\
                0            & \mbox{Otherwise}
            \end{cases}
        \end{equation*}

        \begin{enumerate}[(a)]
            \item Are X and Y independent? (Show the detailed steps)
            \item Calculate E[Y|X>2]
        \end{enumerate}
    }

    \begin{framed}
        \begin{enumerate}[(a)]
            \item By the definition of independence,
                  \[f_{X,Y}(x,y)=f_X(x)f_Y(y)\]
                  The marginals are,
                  \begin{equation*}
                      \begin{split}
                          f_Y(y) & = \int_{0}^{\infty}6e^{-2x-3y}\operatorname{d}x\\
                          & = 6e^{-3y}\int_{0}^{\infty}e^{-2x}\operatorname{d}x\\
                          f_Y(y) & = 3e^{-3y}\\
                          f_X(x) & = \int_{0}^{\infty}6e^{-2x-3y}\operatorname{d}y\\
                          & = 6e^{-2x}\int_{0}^{\infty}e^{-3y}\operatorname{d}y\\
                          f_X(x) & = 2e^{-2x}
                      \end{split}
                  \end{equation*}
                  We can see that,
                  \[f_{X,Y}(x,y)=f_X(x)f_Y(y)\]
                  Hence, $X$ and $Y$ are independent.
            \item \begin{equation*}
                      \begin{split}
                          E[Y|X>2] & = E[Y]\\
                          & = \int_{0}^{\infty}yf_Y(y)\operatorname{d}y\\
                          & = \int_{0}^{\infty}3ye^{-3y}\operatorname{d}y\\
                          & = 3\int_{0}^{\infty}ye^{-3y}\operatorname{d}y\\
                          & = 3\frac{\Gamma(1+1)}{3^{1+1}}\\
                          E[Y|X>2] & = \frac{1}{3}\\
                      \end{split}
                  \end{equation*}
        \end{enumerate}
    \end{framed}

    \break
    \question[30]{
        A defective coin minting machine produces coins whose probability of heads is a random variable P with PDF:
        \begin{equation*}
            f_P(p) = \begin{cases}
                1 + \sin{(2\pi p)} & if \; p \;\in \; [0,1], \\
                0                  & \mbox{Otherwise}
            \end{cases}
        \end{equation*}

        In essence, a specific coin produced by this machine will have a fixed probability $P = p$ of giving heads, but you do not know initially what that probability is. A coin produced by this machine is selected and tossed repeatedly, with successive tosses assumed independent.

        Let A represent that the first coin toss resulted in heads, find the conditional PDF of P i.e. $f_{P|A}(p)$.

        Hint: We have $P(A\;|\;P=p) = p$
    }

    \begin{framed}
        From Bayes rule,
        \begin{equation*}
            \begin{split}
                P(A) & = \int_{P}f_P(p)P(A|P=p)\operatorname{d}p\\
                     & = \int_{0}^{1}(1 + \sin{(2\pi p)})p\operatorname{d}p\\
                     & = \int_{0}^{1}(p + p\sin{(2\pi p)})\operatorname{d}p\\
                     & = \frac{1}{2}+\int_{0}^{1}p\sin{(2\pi p)}\operatorname{d}p\\
                     & = \frac{1}{2}+\int p\sin{(2\pi p)}\operatorname{d}p\bigg|_{0}^{1}\\
                     & = \frac{1}{2}+\left(p\int \sin{(2\pi p)}\operatorname{d}p-\int \operatorname{\frac{d}{\operatorname{d}p}}(p)\int \sin{(2\pi p)}\operatorname{d}p\operatorname{d}p\right)\bigg|_{0}^{1}\\
                     & = \frac{1}{2}+\left(-\frac{p}{2\pi} \cos{(2\pi p)}+\frac{1}{4\pi^2}\sin{(2\pi p)}\right)\bigg|_{0}^{1}\\
                     & = \frac{1}{2}+\left(-\frac{1}{2\pi} \cos{(2\pi)}+\frac{1}{4\pi^2}\sin{(2\pi )}\right)-\left(-\frac{0}{2\pi} \cos{(0)}+\frac{1}{4\pi^2}\sin{(0)}\right)\\
                P(A) & = \frac{1}{2}-\frac{1}{2\pi}\\
                f_{P|A}(p) & = \frac{f_{P}(p)P(A|P=p)}{P(A)}\\
                f_{P|A}(p) & = \frac{\left(1 + \sin{(2\pi p)}\right)p}{\frac{1}{2}-\frac{1}{2\pi}}\\
                f_{P|A}(p) & = \frac{2\pi p\left(1 + \sin{(2\pi p)}\right)}{\pi-1}\\
            \end{split}
        \end{equation*}
    \end{framed}
\end{questions}

\noindent
\hrulefill \; Eid Mubarak in Advance $\ddot \smile$ \; \hrulefill
\end{document}