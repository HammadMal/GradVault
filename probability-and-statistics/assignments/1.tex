%%This is a very basic article template.
%%There is just one section* and two subsections.
\documentclass[a4paper, 11pt]{article}
\usepackage[pdftex]{graphicx}
\usepackage{parskip}
\usepackage{hyperref}
\usepackage[all]{hypcap}
\usepackage{amsmath}
\usepackage{amsfonts}
\usepackage{enumitem}
\usepackage{amssymb}

\title{Habib University\\Math 310 - Introduction to Probability and Statistics\\Assignment 1}
\author{Muhammad Meesum Ali Qazalbash}

\newcommand{\mat}[1]{\boldsymbol { \mathsf{#1}} }

\begin{document}
\setlength{\parskip}{10pt}
\setlength{\parindent}{0pt}
\DeclareGraphicsExtensions{.pdf,.png,.gif,.jpg}
\maketitle


\section*{Question 1}
Consider the five regular dice that come in the shapes of Platonic solids, namely:
a 4-sided die (tetrahedron), a 6-sided die (cube), an 8-sided die (octahedron), a 12-sided die
(dodecahedron),  and a 20-sided die (icosahedron).  Audrey has one die of each type.  She
selects one of the dice at random from her bag of dice (all are equally likely to be chosen),
and when she rolls it, the value 10 appears.  What is the conditional probability that the die
Audrey selected was the dodecahedron?
\subsection*{Solution}

Lets set up the folowing nomenclature,
\begin{align*}
    \text{tetrahedron} & = D_4    & \text{cube}         & = D_6    \\
    \text{octahedron}  & = D_8    & \text{dodecahedron} & = D_{12} \\
    \text{icosahedron} & = D_{20}
\end{align*}
Our sample space is defined to be,
\[\Omega := \{D_{4},D_{6},D_{8},D_{12},D_{20}\}\]
Audrey selects one of the dice at random from her bag of dice (all are equally likely to be chosen),
and when she rolls it, the value 10 appears. Let's call is event $A$. 10 can only appear in $D_{12}$
and $D_{20}$. The probability of $A$ is,
\begin{multline*}
    P(A)=P(D_{4})P(A|D_{4})+P(D_{6})P(A|D_{6})+P(D_{8})P(A|D_{8})\\
    +P(D_{12})P(A|D_{12})+P(D_{20})P(A|D_{20})
\end{multline*}
\[P(A)=\left(\frac{1}{5}\right)(0)+\left(\frac{1}{5}\right)(0)+\left(\frac{1}{5}\right)(0)
    +\left(\frac{1}{5}\right)\left(\frac{1}{12}\right)+\left(\frac{1}{5}\right)\left(\frac{1}{20}\right)\]
\[P(A)=\frac{2}{75}\]
The probability of $B$ given that
$A$ has happened,
\begin{equation*}
    \begin{split}
        P(B|A)&=\frac{P(B \cap A)}{P(A)}\\
        P(B|A)&=\frac{(\frac{1}{5})(\frac{1}{12})}{\frac{2}{75}}\\
        P(B|A)&=\frac{5}{8}=0.625
    \end{split}
\end{equation*}

\section*{Question 2}
Consider a standard deck of 52 cards (13 possible values, 4 suits of each).  Alice and Bob
are each dealt five cards at random, without replacement between cards, and also without
replacement between the two people.
\begin{enumerate}[label=(\alph*)]
    \item In a flush, all 5 cards have the same suit. Given that Alice receives a flush, what is the conditional
          probability that Bob receives a flush too?
    \item In a (new type of) full house, let\'s say that 3 cards have the same suit, and 2 cards have
          the same suit; assume that a flush does not count as a full house, i.e., the two suits must be
          different. Given that Alice receives a full house, what is the conditional probability that Bob
          receives a flush?
\end{enumerate}
\subsection*{Solution}
\begin{enumerate}[label=(\alph*)]
    \item Suppose the Alice takes a flush of any suits. The number of possible choises from this event 13 at out
          of which 5 we have to be chossen. We have 4 choices for suits out of which 1 has to be chossen. As Alice
          takes one flush there are 3 set of suits which are untouched and one had been reduces because Alice took
          flush from it. The conditional probability that Bob receives a flush too will be,
          \begin{multline*}
              P(\text{Bob receives flush}|\text{Alice receives flush})=
              \\P(\text{Bob receives flush from untouced suits}|\text{Alice receives flush})
              \\+P(\text{Bob receives flush from touced suits}|\text{Alice receives flush})
          \end{multline*}
          \[P(\text{Bob receives flush}|\text{Alice receives flush})= \frac{\binom{3}{1}\binom{13}{5}}{\binom{47}{5}}
              +\frac{\binom{1}{1}\binom{8}{5}}{\binom{47}{5}}\]
          \[P(\text{Bob receives flush}|\text{Alice receives flush})\approx0.002553\]
    \item If Alice receives a house then Bob has to choose from 2 untounced suits, one suit from which 2 cards has
          been taken and and another suit from which 3 cards has been taken. The conditional probability that Bob
          receives a flush given that Alice receives a house is,
          \begin{multline*}
              P(\text{Bob receives flush}|\text{Alice receives house})=
              \\P(\text{Bob receives flush from untouced suits}|\text{Alice receives house})
              \\+P(\text{Bob receives flush from 3rd touced suits}|\text{Alice receives house})
              \\+P(\text{Bob receives flush from 4th touced suits}|\text{Alice receives house})
          \end{multline*}
          \begin{multline*}
              P(\text{Bob receives flush}|\text{Alice receives house})=
              \\\frac{\binom{2}{1}\binom{13}{5}}{\binom{47}{5}}
              +\frac{\binom{1}{1}\binom{11}{5}}{\binom{47}{5}}
              +\frac{\binom{1}{1}\binom{10}{5}}{\binom{47}{5}}
          \end{multline*}
          \[P(\text{Bob receives flush}|\text{Alice receives house})\approx0.002143\]

\end{enumerate}

\section*{Question 3}
There are three coins available.  Coin ``A'' is dented in such a way that, when we flip it, we have a probability of 49\% of getting heads.  Coin ``B'' is dented in such a way that, when we flip it, we have a probability of 52\% of getting heads.  Coin ``C'' is a fair coin. We randomly grab a coin (all three coins are equally likely to be selected) and we flip it 7 times
\begin{enumerate}[label=(\alph*)]
    \item If we get 7 heads, what is the conditional probability that we selected coin ``A''
    \item If we get 7 heads, what is the conditional probability that we selected coin ``B''
    \item If we get 7 heads, what is the conditional probability that we selected coin ``C''
\end{enumerate}
\subsection*{Solution}
Given that,
\begin{equation*}
    P(H_A)=0.49 \qquad P(H_B)=0.52 \qquad P(H_C)=0.50
\end{equation*}
The probability of getting 7 heads by Total Probality Theorem is,
\begin{multline*}
    P(7\text{ heads})=P(\text{coin A})P(7\text{ heads}|\text{coin A})
    \\+P(\text{coin B})P(7\text{ heads}|\text{coin B})+P(\text{coin C})P(7\text{ heads}|\text{coin C})
\end{multline*}
\[P(7\text{ heads})=\frac{1}{3}(0.49)^7+\frac{1}{3}(0.52)^7+\frac{1}{3}(0.5)^7\]
\[P(7\text{ heads})\approx 0.008292\]
If we get 7 heads, the conditional probability that we selected coin ``X'' will be according to Baye\'s Rule is,
\[P(7\text{ heads})P(\text{coin }X|7\text{ heads})=P(\text{coin }X)P(7\text{ heads}|\text{coin }X)\]
\[P(\text{coin }X|7\text{ heads})=\frac{P(\text{coin }X)P(7\text{ heads}|\text{coin }X)}{P(7\text{ heads})}\]
\begin{enumerate}[label=(\alph*)]
    \item If $X=A$
          \[\Rightarrow P(\text{coin }A|7\text{ heads})=\frac{P(\text{coin }A)P(7\text{ heads}|\text{coin }A)}{P(7\text{ heads})}\]
          \[\Rightarrow P(\text{coin }A|7\text{ heads})=\frac{\frac{1}{3}(0.49)^7}{0.008292}\approx0.2726\]
    \item If $X=B$
          \[\Rightarrow P(\text{coin }B|7\text{ heads})=\frac{P(\text{coin }B)P(7\text{ heads}|\text{coin }B)}{P(7\text{ heads})}\]
          \[\Rightarrow P(\text{coin }B|7\text{ heads})=\approx \frac{\frac{1}{3}(0.52)^7}{0.008292}\approx 0.4133\]
    \item If $X=C$
          \[\Rightarrow P(\text{coin }C|7\text{ heads})=\frac{P(\text{coin }C)P(7\text{ heads}|\text{coin }C)}{P(7\text{ heads})}\]
          \[\Rightarrow P(\text{coin }C|7\text{ heads})=\frac{\frac{1}{3}(0.50)^7}{0.008292}\approx 0.3140\]
\end{enumerate}


\section*{Question 4}
Suppose that 37\% of the pages of an analysis textbook have a theorem, 29\% of the pages
of  an  algebra  textbook  have  a  theorem,  and  20\%  of  the  pages  of  a  probability  textbook
have a theorem.  The academic advisors have encouraged students to take at most one math
course at a time, and as a result, 5\% of students are taking analysis, 13\% are taking algebra,
10\% are taking probability, and the other 72\% are not taking a math course.
If you randomly walk up to a table at the local coffee shop and peek over the student\'s
shoulder,  and  the  student  happens  to  be  reading  a  theorem  in  a  math  book,  what  is  the
probability that student is enrolled in a probability course?
\subsection*{Solution}
Lets set the following nomenclature,
\begin{align*}
    T=Theorem  &  & A=Analysis              \\
    Al=Algebra &  & \text{Prob}=Probability
\end{align*}
The probabilities are,
\begin{equation*}
    P(T|A)=0.37 \qquad P(T|Al)=0.29 \qquad P(T|\text{Prob})=0.20
\end{equation*}
\begin{equation*}
    P(A)=0.05 \qquad P(Al)=0.13 \qquad P(\text{Prob})=0.10
\end{equation*}
By Total Probability Theorem $P(T)$ will be,
\begin{multline*}
    P(T)=P(A)P(T|A)+P(Al)P(T|Al)+P(\text{Prob})P(T|\text{Prob})\\
    +P(\text{none of the textbook})P(T|\text{none of the textbook})
\end{multline*}
\begin{multline*}
    P(T)=(0.05)(0.37)+(0.13)(0.29)+(0.10)(0.20)\\
    +P(\text{none of the textbook})(0)
\end{multline*}
\[P(T)=0.0762\]
We have to find the probability of a student reading a Probability Book given that the observer is
seeing a Theorem on the opened page.
\begin{equation*}
    \begin{split}
        P(T)P(\text{Prob}|T)&=P(\text{Prob})P(T|\text{Prob})\\
        P(\text{Prob}|T)&=\frac{P(\text{Prob})P(T|\text{Prob})}{P(T)}\\
        P(\text{Prob}|T)&=\frac{(0.10)(0.20)}{0.0762}\\
        P(\text{Prob}|T)&=\frac{100}{381}\approx 0.2625
    \end{split}
\end{equation*}

\section*{Question 5}
We  deal  one  card  at  a  time  from  a traditional  deck  of  52  cards,  with  replacement  of
the card back into the deck and also shuffling in between each deal.  We continue in this
fashion until the first J, Q, or K appears.
\begin{enumerate}[label=(\alph*)]
    \item What is the probability that it takes strictly less than 5 attempts to do this, i.e., that
          we get our first J, Q, or K within the first 4 attempts?
    \item What is the probability that it takes exactly 5 attempts to do this?
    \item Solve 5a again, but without replacing and shuffling the cards in between each deal
    \item Solve 5b again, but without replacing and shuffling the cards in between each deal
\end{enumerate}
\subsection*{Solution}
\begin{enumerate}[label=(\alph*)]
    \item Lets call getting Jack, Queen or King an event $\mathcal{A}$, $n$ is the attempt nunmber
          and $N$ is the total attempts.
          \[P(N<5|\mathcal{A})=1-P(N<5|\neg\mathcal{A})\]
          \[P(N<5|\mathcal{A})=1-\prod_{i=1}^{4}P(n=i^{th}|\neg\mathcal{A})\]
          \[P(N<5|\mathcal{A})=1-\prod_{i=1}^{4}\left(1-P(n=i^{th}|\mathcal{A})\right)\]
          The probability of getting either Jack, Queen or King is same at every place, because we are
          doing with replacement, that is $\frac{12}{52}$.
          \[P(N<5|\mathcal{A})=1-\prod_{i=1}^{4}\left(1-\frac{12}{52}\right)\]
          \[P(N<5|\mathcal{A})=1-\left(\frac{10}{13}\right)^{4}\approx0.6498\]
    \item For exactly 5 attempts the event $\mathcal{A}$ will happen at 5th attempt therefore, all the attempts before
          5th attempt will be a faliure of $\mathcal{A}$.
          \[P(N=5|\mathcal{A})=P(N=5^{th}|\mathcal{A})\prod_{i=1}^{4}P(n=i^{th}|\neg\mathcal{A})\]
          \[P(N=5|\mathcal{A})=P(N=5^{th}|\mathcal{A})\prod_{i=1}^{4}\left(1-P\left(n=i^{th}|\mathcal{A}\right)\right)\]
          \[P(N=5|\mathcal{A})=\left(\frac{12}{52}\right)\prod_{i=1}^{4}\left(1-\frac{12}{52}\right)\]
          \[P(N=5|\mathcal{A})=\left(\frac{12}{52}\right)\left(\frac{42}{52}\right)^{4}\approx 0.08087\]
    \item Now we are doing the same random experiment but with replacement.
          \[P(N<5|\mathcal{A})=1-P(N<5|\neg\mathcal{A})\]
          \[P(N<5|\mathcal{A})=1-\prod_{i=1}^{4}P\left(n=i\bigg|\neg\mathcal{A}\cap \bigcap_{j=1}^{i-1}(n=j)\right)\]
          The probability is trivial to calculate,
          \[P(N<5|\mathcal{A})=1-\prod_{i=1}^{4}\left(1-P\left(n=i\bigg|\mathcal{A}\cap \bigcap_{j=1}^{i-1}(n=j)\right)\right)\]
          \[P(N<5|\mathcal{A})=1-\prod_{i=1}^{4}\left(1-\frac{12}{52-i+1}\right)\]
          \[P(N<5|\mathcal{A})=1-\prod_{i=1}^{4}\left(\frac{41-i}{53-i}\right)\]
          \[P(N<5|\mathcal{A})=1-\left(\frac{40}{52}\right)\left(\frac{39}{51}\right)\left(\frac{38}{50}\right)\left(\frac{37}{49}\right)\]
          \[P(N<5|\mathcal{A})=\frac{2759}{4165}\approx 0.6624\]
    \item For exactly 5 attempts the event $\mathcal{A}$ will happen at 5th attempt therefore, all the attempts before
          5th attempt will be a faliure of $\mathcal{A}$.
          \[P(N=5|\mathcal{A})=P\left(n=5\bigg|\mathcal{A}\cap \bigcap_{j=1}^{4}(n=j)\right)\prod_{i=1}^{4}\left(1-P\left(n=i\bigg|\mathcal{A}\cap \bigcap_{j=1}^{i-1}(n=j)\right)\right)\]
          \[P(N=5|\mathcal{A})=\left(\frac{40}{52}\right)\left(\frac{39}{51}\right)\left(\frac{38}{50}\right)\left(\frac{37}{49}\right)\left(\frac{12}{48}\right)\]
          \[P(N=5|\mathcal{A})=\frac{703}{8330}\approx 0.08439\]
\end{enumerate}


\section*{Question 6}
Roll three (6-sided) dice.
\begin{enumerate}[label=(\alph*)]
    \item Using sets, find the probability that at least one value of ''2'' appears
    \item Using sets, find the probability that at most one value of ''2'' appear
\end{enumerate}
\subsection*{Solution}
The set of all possible outcomes for one die is,
\[\mathcal{X} :=\{1,2,3,4,5,6\}\]
For three six sided dies rolled one by one, the sample space for this random experiment is,
\[\Omega := \{(x_1,x_2,x_3)|x_1,x_2,x_3\in\mathcal{X}\}\]
The probability law associated to this random expreiment is uniform probability for each outcome.
\[|\Omega|=216\Rightarrow P(\{x\})=\frac{1}{216}\forall x\in\Omega\]
\begin{enumerate}[label=(\alph*)]
    \item Lets call an event A where at least one value of ''2'' appears in the roll. The members of the set A will be,
          \begin{multline*}
              A: =\{(2,x,y)|x,y\in\mathcal{X}\}\cup \{(x,2,y)|x,y\in\mathcal{X}\}\\\cup \{(x,y,2)|x,y\in\mathcal{X}\}
          \end{multline*}
          We know that the cardinality of union of three sets is given by,
          \[|P\cup Q\cup R|=|P|+|Q|+|R|-|P\cap Q|-|Q\cap R|-|R\cap P|+|P\cap Q\cap R|\]
          There are exaclty 36 elements in each set, 6 elements common in any two sets that made up the $A$, 1 common element in all of them.
          \[|A|=36+36+36-6-6-6+1\]
          \[|A|=91\]
          \[P(A)=\frac{91}{216}\approx 0.4213\]

    \item Lets call an event B where at most one value of ''2'' appear in the roll. The members of the set A will be,
          \[B :=\{(x,y,z),(2,x,y),(x,2,y),(x,y,2)|x,y,z\in\mathcal{X}/\{2\}\}\]
          \[\Rightarrow |B|=200\]
          The cardinalty of these sets is trivial to find. We have 5 choices for $x$ and 5 choices for $y$ because 2 has already
          appeared in the roll and we want atmost one 2 in three rolls, and there are 3 such tuples that means there are
          $5 \times 5 \times 3$ and $5\times 5\times 5$ when no 2 appeared, different elements in set $B$. Note that the tuples
          are disjoint, the above proposition is only applicable at disjoint objects. The probability of event $B$ is,
          \[P(B)=\frac{200}{216}\approx 0.9259\]
\end{enumerate}


\end{document}