\documentclass[a4paper]{exam}

\usepackage{amsmath,amssymb}
\usepackage{geometry}
\usepackage{graphicx}
\usepackage{hyperref}
\usepackage{titling}

% Header and footer.
\pagestyle{headandfoot}
\runningheadrule
\runningfootrule
\runningheader{CS 212, Fall 2022}{WC 10: Complexity Classes}{\theauthor}
\runningfooter{}{Page \thepage\ of \numpages}{}
\firstpageheader{}{}{}

\printanswers

\title{Weekly Challenge 10: Complexity Classes}
\author{mq06861} % <=== replace with your student ID, e.g. xy012345
\date{CS 212 Nature of Computation\\Habib University\\Fall 2022}

\qformat{{\large\bf \thequestion. \thequestiontitle}\hfill}
\boxedpoints

\begin{document}
\maketitle

\begin{questions}
  
\titledquestion{Checking Primality}

Explain succinctly why the language recognized by the following Turing Machine, M, does not belong to $\mathbf{P}$ (assume the input is represented in binary)

$M = $ On input $n$:
\begin{enumerate}
\item Check if 2 divides $n$, if so \textit{Reject}.
\item Otherwise, repeat step 1 for all numbers less than $n$. That is, check if 3 divides $n$. If so \textit{Reject}, otherwise check if 4 divides $n$, if so Reject, and so on.
\item If all numbers less than $n$ have been checked, \textit{Accept}.  
\end{enumerate}
	\begin{solution}
		We know that,
		\[
			\displaystyle\mathbf{P} := \bigcup_{k} \operatorname{TIME}(n^k)
		\]
		where \(n\) is the size of the input. Let \(p = L(M)\),
		\(p\) is the set of all prime numbers represented in binary.
		On \(M\) input \(n\) would be the binary representation of
		some number \(p'\). Where the size of input \(n\) would be the
		size of the binary representation of \(p'\), which would be some
		\[q = \lfloor\log_2(p')\rfloor+1\]
		Assuming the time for step 1 of algorithm is \(O(1)\), then on
		input \(n\) of size \(q\), \(M\) takes \(p'\) steps where \(p'\)
		is th number that \(n\) is binary represntation of. So for
		input \(n\) of size \(q\) the \(\operatorname{TIME}(M)\) = \(O(2^{q-1})\).
		Which make the the time of \(M\) be exponential.
		\[L(M)\notin\mathbf{P}\]
  	\end{solution}
\end{questions}
\end{document}

%%% Local Variables:
%%% mode: latex
%%% TeX-master: t
%%% End:
