\documentclass[a4paper]{exam}

\usepackage{amsmath}
\usepackage{geometry}
\usepackage{graphicx}
\usepackage{hyperref}
\usepackage{titling}
\usepackage{amssymb}

% Header and footer.
\pagestyle{headandfoot}
\runningheadrule
\runningfootrule
\runningheader{CS 212, Fall 2022}{WC 06: Regular and Context-Free Languages}{\theauthor}
\runningfooter{}{Page \thepage\ of \numpages}{}
\firstpageheader{}{}{}

\printanswers

\title{Weekly Challenge 06: Regular and Context-Free Languages}
\author{mq06861} % <=== replace with your student ID, e.g. xy012345
\date{CS 212 Nature of Computation\\Habib University\\Fall 2022}

\qformat{{\large\bf \thequestion. \thequestiontitle}\hfill}
\boxedpoints

\begin{document}
\maketitle

\begin{questions}

	\titledquestion{A PDA for every RE}

	Show that for any given regular expression, there exists a pushdown automaton that recognizes the generated language.

	\begin{solution}
		If \(R\) is the regular expression and then let \(M\) be an DFA that accepts it. \(M\) is defined as \((Q,\Sigma, \delta,q_0,F)\), where \(Q\) is the set of states, \(\Sigma\) is the input alphabet, \(\delta\) is the transition function, \(q_0\) is the start state, and \(F\) is the set of final states. Let \(M' = (Q',\Sigma',\Gamma,\delta',q_0',F')\) be a PDA that accepts \(L(M)\).
		
		We can construct a PDA for every DFA by simply conserving every thing in the DFA except the transition function. Transition function of PDA includes which operation to perform on the stack, but in case of DFA we do not need any stack therefore we would not perfrom any operation on stack.
		
		Therefore the construction of \(M'\) is as follows,
		\begin{enumerate}
			\item \(Q' = Q\)
			\item \(\Sigma'=\Sigma\)
			\item \(\Gamma=\{\varepsilon\}\)
			\item \(\forall a\in\Sigma'_{\varepsilon},\forall q\in Q',\delta'(q,a,\varepsilon) = \{(\delta(q,a),\varepsilon)\}\)
			\item \(q_0'=q_0\)
			\item \(F'=F\)
		\end{enumerate}
		\(M'\) will accepts \(L(M)\) that is generated by \(R\). Hence for every regular expression there exists a pushdown automaton that recognizes the generated language.\hfill\(\blacksquare\)
	\end{solution}
	\end{questions}
\end{document}

%%% Local Variables:
%%% mode: latex
%%% TeX-master: t
%%% End:
