\documentclass[a4paper]{exam}

\usepackage{amsmath}
\usepackage{geometry}
\usepackage{graphicx}
\usepackage{hyperref}
\usepackage{amssymb}

\printanswers

\title{Weekly Challenge 02: Equivalence of Finite Automata}
\author{CS 212 Nature of Computation\\Habib University}
\date{Fall 2022}

\qformat{{\large\bf \thequestion. \thequestiontitle}\hfill}
\boxedpoints

\begin{document}
\maketitle

\begin{questions}

	\titledquestion{NFA-DFA Equivalence}

	Theorem 1.39 in our textbook states that, ``\emph{Every nondeterministic finite automaton has an equivalent deterministic finite automaton}'', and then provides a proof by construction.

	Prove that the DFA obtained from an NFA by applying the given construction is indeed equivalent. That is, show that the constructed DFA accepts the same language as the given NFA and vice versa.

	You are expected to submit an original proof (k.e. developed by you) that is correct and exhibits sound and precise argumentation. If you consult any sources for guidance, make sure to cite and acknowledge them duly. Your \LaTeX\ file should compile without errors on the instructors' machines. If you still use Overleaf, make sure that there are no warnings or errors. Files that do not compile cannot be graded.


	\begin{solution}
		Suppose we have an DFA obtained out of NFA defined as,
		\begin{align}
			N=(Q,\Sigma, \delta, q_0, F) &  & D=(Q',\Sigma', \delta', q'_0, F')
		\end{align}
		Where,
		\begin{equation*}
			\begin{split}
				Q&=\mathcal{P}(Q')\\
				\Sigma&=\Sigma'\cup\{\varepsilon\}\\
				\delta(R,a)&=\bigcup_{r\in R}E(\delta'(r,a))\\
				q_0&=E(\{q'_0\})\\
				F&=\{R|R\in Q, \exists r\in R \ni r\in F\}\\
				E(S)&=\{\varepsilon\text{-closure of set }S\}
			\end{split}
		\end{equation*}
		We have to prove that,
		\[(N\longleftrightarrow D)\iff (L(N)=L(D))\]
		We can prove that \(L(N)\subseteq L(D)\) with induction. Suppose a string \(w\in L(N)\) such that \(w=w_1w_2w_3\cdots w_n\). \(w\) passes thorugh series of stages lets call them \(r=r_0r_1r_2\cdots r_n\) such that \(r_0=q_0\), \(r_i\in Q\) and \(r_n\in F\). To pass the same string in \(D\) we would have to pass it through different stages lets call them \(r'=r'_0r'_1r'_2\cdots r'_n\) such that \(r_0'=q_0'\), \(r_i'\in Q'\) and \(r_n'\in F'\).

		\textbf{Base Case}

		We know that \(r_0=q_0=E(\{q_0'\})\), then \(r_1\in\delta(r_0,w_1)\),
		\begin{equation*}
			\begin{split}
				r_1'\in&\delta(r_0,w_1)\\
				r_1'\in&\bigcup_{r\in R}E(\delta'(r,w_1))\\
				r_1'\in&E(\delta'(r_0,w_1))\\
				r_1'\in&E(\delta'(E(\{q_0'\}),w_1))\\
				r_1'\in&r_1
			\end{split}
		\end{equation*}

		\textbf{Hypothsise}

		We hypothsize that for \(w_k\) for some \(k\le n\), \(r_{k-1}'\in r_{k-1}\). This could easily be translated to english language as for the \(k^{th}\) element of the string \(w\), we have \(r_{k-1}\) stage in NFA that has a corresponding stage in DFA as \(r_{k-1}'\).

		\textbf{Inductive Step}

		We have to prove that for \(w_{k+1}\), \(r_k'\in r_k\). We know that the string till \(w_{k}\) is accepted by the \(N\). Therefore,
		\[r_k'=\delta'(r_{k-1}',w_k)\]
		\[r_k\in\delta(r_{k-1},w_k)\]
		According to the definiton of transition function for NFA.
		\begin{equation*}
			\begin{split}
				r_k&\in\bigcup_{r\in r_{k-1}}E(\delta'(r,w_k))\\
				&\bigvee_{r\in r_{k-1}}r_k\in E(\delta'(r,w_k))
			\end{split}
		\end{equation*}
		Beside the \(\varepsilon\) transition we can choose \(r_k'\) from \(r_k\). This means \(r_k'\in r_k\). Hence this is true for all integral values of \(k\). Which means for a DFA derived from NFA (\(N\rightarrow D\)), the language of NFA has all the elemets that are elements of the language of DFA meaning it contain all the strings that would be acceptable by DFA.

		Conversely,

		We can to prove that \(L(N)\supseteq L(D)\). For \(w_{k+1}\), \(r_k'\in r_k\). We know that the string till \(w_{k}\) is accepted by the \(D\). Therefore,
		\[r_k'=\delta'(r_{k-1}',w_k)\]
		\[r_k\in\delta(r_{k-1},w_k)\]
		By using the definiton of transition function for NFA.
		\begin{equation*}
			\begin{split}
				r_k&\in\bigcup_{r\in r_{k-1}}E(\delta'(r,w_k))\\
				&\bigvee_{r\in r_{k-1}}r_k\in E(\delta'(r,w_k))
			\end{split}
		\end{equation*}

		Beside the \(\varepsilon\) transition we can choose \(r_k'\) from \(r_k\). This means \(r_k'\in r_k\). Hence this is true for all integral values of \(k\). Which means for a NFA derived from DFA (\(N\rightarrow D\)), the language of DFA has all the elemets that are elements of the language of NFA meaning it contain all the strings that would be acceptable by NFA.

		We have proved above that for DFA obtained from NFA, language of \(L(N)\subseteq L(D)\). And for the NFA derived from DFA, language of \(L(N)\supseteq L(D)\). Therefore \(L(N)=L(D\). Hence DFA obtained from an NFA by applying the given construction is indeed equivalent. \(\blacksquare\)
	\end{solution}
\end{questions}
\end{document}