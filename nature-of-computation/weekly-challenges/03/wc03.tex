\documentclass[a4paper]{exam}

\usepackage{amsmath}
\usepackage{amssymb}
\usepackage{geometry}
\usepackage{graphicx}
\usepackage{hyperref}
\usepackage{titling}

% Header and footer.
\pagestyle{headandfoot}
\runningheadrule
\runningfootrule
\runningheader{CS 212, Fall 2022}{WC 03: Checking Regularity}{\theauthor}
\runningfooter{}{Page \thepage\ of \numpages}{}
\firstpageheader{}{}{}

\printanswers

\title{Weekly Challenge 03: Checking Regularity}
\author{mq06861} % <=== replace with your student ID, e.g. xy012345
\date{CS 212 Nature of Computation\\Habib University\\Fall 2022}

\qformat{{\large\bf \thequestion. \thequestiontitle}\hfill}
\boxedpoints

\begin{document}
\maketitle

\begin{questions}

	\titledquestion{Mystery Language}

	Is the following language regular or not? Provide a proof to justify your claim.
	\[L = \{ w^iw^j \mid w\in\{0,1\}^*, 0 \leq i \leq j \}\]

	\begin{solution}
		We can prove that \(L\) is equals to \(\Sigma^{*}\) by set association.
		\[L\subseteq \Sigma^{*}\]
		This is trivial because every word in \(L\) is derived using concatination, and \(\Sigma^{*}\) is closed under concatination.
		\begin{align*}
			L                   & \supseteq \Sigma^{*} \\
			\iff w\in\Sigma^{*} & \implies w\in L
		\end{align*}
		We can write \(w=w^{0}w^{1}\) which is the form of words in \(L\).
		\begin{align*}
			(L\subseteq\Sigma^{*})\land(L\supseteq\Sigma^{*})\iff (L=\Sigma^{*})
		\end{align*}
		We have shown that language \(L\) is indeed equals to \(\Sigma^{*}\) and \(\Sigma^{*}\) is a regular language. Therefore \(L\) is a regular language.
		\center\(\blacksquare\)
	\end{solution}
\end{questions}
\end{document}

%%% Local Variables:
%%% mode: latex
%%% TeX-master: t
%%% End:
