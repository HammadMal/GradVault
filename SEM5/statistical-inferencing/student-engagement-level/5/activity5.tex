%%%%%%%%%%%%%%%%%%%%%%%%%%%%% Define Article %%%%%%%%%%%%%%%%%%%%%%%%%%%%%%%%%%
\documentclass{exam}
%%%%%%%%%%%%%%%%%%%%%%%%%%%%%%%%%%%%%%%%%%%%%%%%%%%%%%%%%%%%%%%%%%%%%%%%%%%%%%%

%%%%%%%%%%%%%%%%%%%%%%%%%%%%% Using Packages %%%%%%%%%%%%%%%%%%%%%%%%%%%%%%%%%%
\usepackage{geometry}
\usepackage{graphicx}
\usepackage{amssymb}
\usepackage{amsmath}
\usepackage{amsthm}
\usepackage{empheq}
\usepackage{mdframed}
\usepackage{booktabs}
\usepackage{lipsum}
\usepackage{graphicx}
\usepackage{color}
\usepackage{psfrag}
\usepackage{pgfplots}
\usepackage{bm}
\usepackage{hyperref}
%%%%%%%%%%%%%%%%%%%%%%%%%%%%%%%%%%%%%%%%%%%%%%%%%%%%%%%%%%%%%%%%%%%%%%%%%%%%%%%

% Other Settings

%%%%%%%%%%%%%%%%%%%%%%%%%% Page Setting %%%%%%%%%%%%%%%%%%%%%%%%%%%%%%%%%%%%%%%
\geometry{a4paper}

%%%%%%%%%%%%%%%%%%%%%%%%%% Define some useful colors %%%%%%%%%%%%%%%%%%%%%%%%%%
\definecolor{ocre}{RGB}{243,102,25}
\definecolor{mygray}{RGB}{243,243,244}
\definecolor{deepGreen}{RGB}{26,111,0}
\definecolor{shallowGreen}{RGB}{235,255,255}
\definecolor{deepBlue}{RGB}{61,124,222}
\definecolor{shallowBlue}{RGB}{235,249,255}
%%%%%%%%%%%%%%%%%%%%%%%%%%%%%%%%%%%%%%%%%%%%%%%%%%%%%%%%%%%%%%%%%%%%%%%%%%%%%%%

%%%%%%%%%%%%%%%%%%%%%%%%%% Define an orangebox command %%%%%%%%%%%%%%%%%%%%%%%%
\newcommand\orangebox[1]{\fcolorbox{ocre}{mygray}{\hspace{1em}#1\hspace{1em}}}
%%%%%%%%%%%%%%%%%%%%%%%%%%%%%%%%%%%%%%%%%%%%%%%%%%%%%%%%%%%%%%%%%%%%%%%%%%%%%%%

%%%%%%%%%%%%%%%%%%%%%%%%%%%% English Environments %%%%%%%%%%%%%%%%%%%%%%%%%%%%%
\newtheoremstyle{mytheoremstyle}{3pt}{3pt}{\normalfont}{0cm}{\rmfamily\bfseries}{}{1em}{{\color{black}\thmname{#1}~\thmnumber{#2}}\thmnote{\,--\,#3}}
\newtheoremstyle{myproblemstyle}{3pt}{3pt}{\normalfont}{0cm}{\rmfamily\bfseries}{}{1em}{{\color{black}\thmname{#1}~\thmnumber{#2}}\thmnote{\,--\,#3}}
\theoremstyle{mytheoremstyle}
\newmdtheoremenv[linewidth=1pt,backgroundcolor=shallowGreen,linecolor=deepGreen,leftmargin=0pt,innerleftmargin=20pt,innerrightmargin=20pt,]{theorem}{Theorem}[section]
\theoremstyle{mytheoremstyle}
\newmdtheoremenv[linewidth=1pt,backgroundcolor=shallowBlue,linecolor=deepBlue,leftmargin=0pt,innerleftmargin=20pt,innerrightmargin=20pt,]{definition}{Definition}[section]
\theoremstyle{myproblemstyle}
\newmdtheoremenv[linecolor=black,leftmargin=0pt,innerleftmargin=10pt,innerrightmargin=10pt,]{problem}{Problem}[section]
%%%%%%%%%%%%%%%%%%%%%%%%%%%%%%%%%%%%%%%%%%%%%%%%%%%%%%%%%%%%%%%%%%%%%%%%%%%%%%%

%%%%%%%%%%%%%%%%%%%%%%%%%%%%%%% Plotting Settings %%%%%%%%%%%%%%%%%%%%%%%%%%%%%
\usepgfplotslibrary{colorbrewer}
\pgfplotsset{width=8cm,compat=1.9}
%%%%%%%%%%%%%%%%%%%%%%%%%%%%%%%%%%%%%%%%%%%%%%%%%%%%%%%%%%%%%%%%%%%%%%%%%%%%%%%

%%%%%%%%%%%%%%%%%%%%%%%%%%%%%%% Title & Author %%%%%%%%%%%%%%%%%%%%%%%%%%%%%%%%
\title{SEL Activity 5}
\author{Muhammad Meesum Ali Qazalbash}
%%%%%%%%%%%%%%%%%%%%%%%%%%%%%%%%%%%%%%%%%%%%%%%%%%%%%%%%%%%%%%%%%%%%%%%%%%%%%%%
\printanswers
\begin{document}
\maketitle

\begin{questions}
    \question Prove that \(\displaystyle\operatorname{E}\left[\left(T-\mu_{T}\right)^{2}\right]=\frac{4A^{3}+2A^{2}}{N}\).

    \begin{solution}
        \begin{equation}\label{proof}
            \begin{split}
                \operatorname{E}\left[\left(T-\mu_{T}\right)^{2}\right]&=\operatorname{var}(T)\\
                &=\operatorname{var}\left(\frac{1}{N}\sum_{i=0}^{N}X^{2}_{i}\right)\\
                &=\frac{1}{N^2}\operatorname{var}\left(\sum_{i=0}^{N}X^{2}_{i}\right)\\
                &=\frac{1}{N^2}\sum_{i=0}^{N}\operatorname{var}\left(X^{2}_{i}\right)\\
                &=\frac{N}{N^2}\operatorname{var}\left(X^{2}\right)\\
                &=\frac{1}{N}\operatorname{var}\left(X^{2}\right)\\
            \end{split}
        \end{equation}
        Where \(X\sim\mathcal{N}(x;A,A)\), therefore the variance of \(X^2\) would be,
        \begin{equation*}
            \begin{split}
                \operatorname{var}\left(X^{2}\right)&=\operatorname{E}\left[X^{4}\right]-\operatorname{E}\left[X^2\right]^2\\
                &=\operatorname{E}\left[X^{4}\right]-\left(A^2+A\right)^2\\
            \end{split}
        \end{equation*}
        We can use the moment generating function of Guassian Randome Variable to calculate the 4th moment,
        \[X\sim \mathcal{N}(\mu,\sigma^{2}) \implies M_{X}(0)=\exp{\left(t\mu+\frac{1}{2}\sigma^2t^2\right)}\]
        We know that,
        \[y=e^{f(t)}\Rightarrow y''''=\left(f''''(0)+4f'''(0)f'(0)+3(f''(0))^2+6f''(0)(f'(0))^2+(f'(0))^{4}\right)e^{f(0)}\]
        Let \(f(t)=\mu t+\frac{1}{2}\sigma^{2}t^{2}\),

        \begin{align*}
            f^{(0)}(t)=\mu t+\frac{1}{2}\sigma^{2}t^{2} & \implies f^{(0)}(0)= 0          \\
            f^{(1)}(t)=\mu+\sigma^{2}t                  & \implies f^{(1)}(0)= \mu        \\
            f^{(2)}(t)=\sigma^{2}                       & \implies f^{(2)}(0)= \sigma^{2} \\
            f^{(3)}(t)=0                                & \implies f^{(3)}(0)= 0          \\
            f^{(4)}(t)=0                                & \implies f^{(4)}(0)= 0          \\
        \end{align*}

        The fourth moment would be,

        \begin{equation*}
            \begin{split}
                \mu_4&=M^{(4)}_{X}(0)\\
                &=\frac{\operatorname{d}^{4}}{\operatorname{d}t^{4}}\left(\exp{\left(t\mu+\frac{1}{2}\sigma^2t^2\right)}\right)\\
                &=\left(0+4(0)(\mu)+3\left(\sigma^{2}\right)^2+6\sigma^{2}(\mu)^2+(\mu)^{4}\right)e^{0}\\
                &=3\sigma^{4}+6\sigma^{2}\mu^2+\mu^{4}\\
                &=3A^{2}+6A^3+A^{4}\\
            \end{split}
        \end{equation*}
        The variance of \(X^2\) would be,
        \begin{equation*}
            \begin{split}
                \operatorname{var}\left(X^{2}\right)&=3A^{2}+6A^3+A^{4}-\left(A^2+A\right)^2\\
                &=3A^{2}+6A^3+A^{4}-A^{4}-2A^{3}-A^{2}\\
                &=4A^{3}+2A^{2}\\
            \end{split}
        \end{equation*}
        Putting the above result in \eqref{proof}, we get,
        \[\operatorname{var}(T)=\operatorname{E}\left[\left(T-\mu_{T}\right)^{2}\right]=\frac{4A^{3}+2A^{2}}{N}\]
        \center \(\blacksquare\)
    \end{solution}

    \newpage

    \question Prove the following claims,
    \begin{enumerate}
        \item \(g(\mu_{T})=A\)
        \item \(\displaystyle g'(\mu_{T})=\frac{1}{2}\left(A+\frac{1}{2}\right)^{-1}\)
        \item \(\displaystyle g''(\mu_{T})=-\frac{1}{4}\left(A+\frac{1}{2}\right)^{-3}\)
    \end{enumerate}

    \begin{solution}
        We know that, \(\mu_{T}=A^{2}+A\) and \(\displaystyle g(T)=-\frac{1}{2}+\sqrt{T+\frac{1}{4}}\)
        \[\therefore g(\mu_{T})=-\frac{1}{2}+\left(A^{2}+A+\frac{1}{4}\right)^{\frac{1}{2}}=-\frac{1}{2}+A+\frac{1}{2}=A\]
        \[\therefore g'(\mu_{T})=\frac{\operatorname{d}g(T)}{\operatorname{d}T}\bigg|_{T=\mu_{T}}=\frac{1}{2}\left(A^{2}+A+\frac{1}{4}\right)^{-\frac{1}{2}}=\frac{1}{2}\left(A+\frac{1}{2}\right)^{-1}\]
        \[\therefore g''(\mu_{T})=\frac{\operatorname{d}^{2}g(T)}{\operatorname{d}T^{2}}\bigg|_{T=\mu_{T}}=-\frac{1}{4}\left(A^{2}+A+\frac{1}{4}\right)^{-\frac{3}{2}}=-\frac{1}{4}\left(A+\frac{1}{2}\right)^{-3}\]
        \center \(\blacksquare\)
    \end{solution}

    \question Using (1), (2) and \(\displaystyle\operatorname{var}\left(\hat{A}\right)=\sigma^{2}_{\hat{A}}\approx\sigma^{2}_{T}\left[g'\left(\mu_{T}\right)\right]^{2}\) to obtain the expression for \(\operatorname{E}\left[\hat{A}\right]\)

    \begin{solution}
        \begin{align*}
            \operatorname{E}\left[\hat{A}\right] & =g(\mu_{T})+\frac{1}{2}g''(\mu_{T})\operatorname{E}\left[(T-\mu_{T})^{2}\right]                            \\
                                                 & =A+\frac{1}{2}\left(-\frac{1}{4}\left(A+\frac{1}{2}\right)^{-3}\right)\left(\frac{4A^{3}+2A^{2}}{N}\right) \\
                                                 & =A-\frac{2A^{2}}{(2A^{2}+1)^{2}}
        \end{align*}
    \end{solution}

    \newpage

    \question Show that \(\displaystyle\operatorname{var}\left(\hat{A}\right)=\frac{A^2}{N\left(A+\frac{1}{2}\right)}\).

    \begin{solution}
        \begin{align*}
            \operatorname{var}\left(\hat{A}\right)=\sigma^{2}_{\hat{A}} & \approx\sigma^{2}_{T}\left[g'\left(\mu_{T}\right)\right]^{2}                              \\
                                                                        & \approx\frac{4A^{3}+2A^{2}}{N}\left[\frac{1}{2}\left(A+\frac{1}{2}\right)^{-1}\right]^{2} \\
                                                                        & \approx4A^{2}\frac{A+\frac{1}{2}}{N}\frac{1}{4}\left(A+\frac{1}{2}\right)^{-2}            \\
                                                                        & \approx\frac{A^2}{N\left(A+\frac{1}{2}\right)}
        \end{align*}
        \center \(\blacksquare\)
    \end{solution}
\end{questions}

\end{document}