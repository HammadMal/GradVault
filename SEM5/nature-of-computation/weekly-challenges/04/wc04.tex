\documentclass[a4paper]{exam}

\usepackage{amsmath}
\usepackage{amssymb}
\usepackage{geometry}
\usepackage{graphicx}
\usepackage{hyperref}
\usepackage{titling}

% Header and footer.
\pagestyle{headandfoot}
\runningheadrule
\runningfootrule
\runningheader{CS 212, Fall 2022}{WC 04: Regular and Context-Free Languages}{\theauthor}
\runningfooter{}{Page \thepage\ of \numpages}{}
\firstpageheader{}{}{}

\printanswers

\title{Weekly Challenge 04: Regular and Context-Free Languages}
\author{mq06861} % <=== replace with your student ID, e.g. xy012345
\date{CS 212 Nature of Computation\\Habib University\\Fall 2022}

\qformat{{\large\bf \thequestion. \thequestiontitle}\hfill}
\boxedpoints

\begin{document}
\maketitle
\begin{questions}

	\titledquestion{Superset}

	Prove that the class of context-free languages is a superset of the class of regular languages.

	\begin{solution}
		% Enter your solution here.
		Let \(\mathcal{R}\) be the class of Regular Languages and \(\mathcal{C}\) be class of Context Free Langauges. We need to show that \[\mathcal{R}\subset\mathcal{C}\] i.e. every regular language is also a context free language. Let \(\mathcal{L}\in\mathcal{R}\) be a regular language. We need to show that \(\mathcal{L}\in\mathcal{C}\). We can construct a context free grammar \(\mathcal{G}\) for \(\mathcal{L}\) as follows:

		\begin{enumerate}
			\item Let \(\mathcal{V}:=\{\mathcal{S}\}\cup\{\mathcal{A}_i\mid i\in\mathbb{N}\}\) be the set of non-terminals.
			\item Let \(\mathcal{T}:=\{\varepsilon\}\cup\mathcal{L}\) be the set of terminals.
			\item Let \(\mathcal{P}:=\{\mathcal{S}\rightarrow \mathcal{A}_i\mathcal{A}_j\mid i,j\in\mathbb{N}\}\cup\{\mathcal{A}_i\rightarrow\sigma\mid\sigma\in\mathcal{L}\land i\in\mathbb{N}\}\) be the set of productions.
			\item Let \(\mathcal{S}\in\mathcal{V}\) be the starting variable.
		\end{enumerate}

		We can see that \(\mathcal{G}\) is a context free grammar for \(\mathcal{L}\). Hence, \(\mathcal{L}\in\mathcal{C}\). Therefore, \(\mathcal{R}\subset\mathcal{C}\).\hfill\(\blacksquare\)
	\end{solution}
\end{questions}
\end{document}

%%% Local Variables:
%%% mode: latex
%%% TeX-master: t
%%% End:
